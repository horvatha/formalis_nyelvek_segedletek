\documentclass[
    %draft,
    ignorenonframetext
    ]{beamer}

\usepackage[T1]{fontenc}
\usepackage[utf8x]{inputenc}
\usepackage{inputenc}
\usepackage[magyar]{babel}
\usepackage{ucs}
\usepackage{tabularx}
\usepackage{fancyvrb}
\usepackage{graphicx}

\usepackage{tikz}
\usetikzlibrary{arrows}
\tikzset{ %inner sep = 0.5mm,
  %minimum size = 5mm,
  {selected path/.style} ={draw,opacity=.5,line width=3pt,green},
  {hide path/.style} ={selected path,white,opacity=.9},
  {base point/.style} ={circle,draw=blue!60,fill=blue!30,inner sep=0},
  {small point/.style} ={base point, minimum size= 1mm},
  {middle point/.style} ={base point, minimum size= 3mm},
  {large point/.style} ={base point, minimum size= 10mm},
  {future point/.style} ={circle,color=white,draw},
  point/.style ={circle,draw=black!60,fill=black!20},
  {new point/.style} ={point,fill=blue!30,draw=blue!60},
  {edge/.style} ={thick,draw},
  {new edge/.style} ={draw,thick,-, blue!50},
  {box/.style}={rectangle,  minimum height=1cm,
                anchor=south west, draw, rectangle,
                fill=yellow!20,text centered, text width= 4cm},
  >=latex,
  state/.style = {thick, circle, draw, color = black,
                  fill = red!20, minimum size = 6mm},
  finalstate/.style = {state, fill=green!20, double},
  automat/.style = {thick, draw, color = black, fill = #1!12,
                  minimum size = 10mm},
  automat/.default = black,
  point/.style ={circle,draw=blue!60,fill=blue!30,
                  inner sep=0, minimum size= 1mm},
 }
\usepackage{pgfplots}
\usepackage{pgfplotstable}
\newcommand{\film}{
\begin{tikzpicture}[rotate=90,scale=.6]
    \draw[fill=black!70] (0,0) -- (0,1) -- (1,1) -- (1,0) -- cycle;
    \draw[fill=black!30] (0.3,0.1) -- (0.3,.9) -- (.7,.9) -- (.7,.1) -- cycle;
    \foreach \i in {.15,.35,...,.76}{
      \draw[fill=white] (0.1,\i) -- (0.1,\i+.1) -- (.2, \i+.1) -- (0.2,\i) -- cycle;
      \draw[fill=white] (0.8,\i) -- (0.8,\i+.1) -- (.9, \i+.1) -- (0.9,\i) -- cycle;
      }
\end{tikzpicture}
}



\mode<article>
{
  \usepackage{fullpage}
  \usepackage{pgf}
  \usepackage[pdftex,colorlinks=true,linkcolor=blue,urlcolor=blue]{hyperref}
  \usepackage{hyperref}
  \usepackage{fancyvrb}
  \setjobnamebeamerversion{python.beamer}
  \newtheorem{definicio}{defin\'ici\'o}
\newcommand{\lecke}[2][90]{\section{ lecke -- #2}

    Tervezett időtartam #1 perc.

}

\newcommand{\coloruncover}[2]{#2}
}

\mode<presentation>
{
%  \usetheme{Dresden}
%  \usetheme{Warsaw}
  \usetheme{Montpellier}
  \usecolortheme{rose}
%  \usepackage{beamerthemebars}

%  \setbeamercovered{transparent}
  %\setbeamertemplate{headline}{}
  \hypersetup{
  pdftitle={Horvath Arpad: Bevezetes a halozatok vizsgalataba},
  unicode
  }
  \newtheorem{definicio}{Defin\'ici\'o}
\newcommand{\lecke}[2][90]{\section{#2}

\begin{frame}
    {Lecke: #2}

    \vfill
    Tervezett időtartam #1 perc.

    \vfill
\end{frame}
}
\definecolor{uncovercolor}{rgb}{1,0.8,0.8}
\newcommand{\coloruncover}[2]{\temporal<#1>{\colorbox{bg}{\phantom{#2}}}{\colorbox{uncovercolor}{\phantom{#2}}}{\colorbox{bg}{#2}}}

}

\newcommand{\oktato}{\textsc{\bfseries\Large\color{green!70!black}PyOkt} }


\newcommand{\Smiley}[1]{%
\begin{tikzpicture}[scale=0.2]
    \newcommand*{\SmileyRadius}{1.0}%
    \draw [fill=brown!10] (0,0) circle (\SmileyRadius)% outside circle
        %node [yshift=-0.22*\SmileyRadius cm] {\tiny #1}% uncomment this to see the smile factor
        ;  

    \pgfmathsetmacro{\eyeX}{0.5*\SmileyRadius*cos(30)}
    \pgfmathsetmacro{\eyeY}{0.5*\SmileyRadius*sin(30)}
    \draw [fill=cyan,draw=none] (\eyeX,\eyeY) circle (0.15cm);
    \draw [fill=cyan,draw=none] (-\eyeX,\eyeY) circle (0.15cm);

    \pgfmathsetmacro{\xScale}{2*\eyeX/180}
    \pgfmathsetmacro{\yScale}{1.0*\eyeY}
    \draw[color=red, domain=-\eyeX:\eyeX]   
        plot ({\x},{
            -0.1+#1*0.15 % shift the smiley as smile decreases
            -#1*1.75*\yScale*(sin((\x+\eyeX)/\xScale))-\eyeY});
\end{tikzpicture}%
}%

\newcommand{\utolsooldal}[1][.]{
\begin{frame}
    {Gratulálunk!}

    Ezzel a modul leckéit sikeresen teljesítette.

    \vfill
    Ami még modulzáró feladatként hátra van:\\
    a modulzáró teszt megoldása#1
\end{frame}
}
%pygments-hez
\usepackage{fancyvrb}
\usepackage{color}

\makeatletter
\def\PY@reset{\let\PY@it=\relax \let\PY@bf=\relax%
    \let\PY@ul=\relax \let\PY@tc=\relax%
    \let\PY@bc=\relax \let\PY@ff=\relax}
\def\PY@tok#1{\csname PY@tok@#1\endcsname}
\def\PY@toks#1+{\ifx\relax#1\empty\else%
    \PY@tok{#1}\expandafter\PY@toks\fi}
\def\PY@do#1{\PY@bc{\PY@tc{\PY@ul{%
    \PY@it{\PY@bf{\PY@ff{#1}}}}}}}
\def\PY#1#2{\PY@reset\PY@toks#1+\relax+\PY@do{#2}}

\def\PY@tok@gd{\def\PY@tc##1{\textcolor[rgb]{0.63,0.00,0.00}{##1}}}
\def\PY@tok@gu{\let\PY@bf=\textbf\def\PY@tc##1{\textcolor[rgb]{0.50,0.00,0.50}{##1}}}
\def\PY@tok@gt{\def\PY@tc##1{\textcolor[rgb]{0.00,0.25,0.82}{##1}}}
\def\PY@tok@gs{\let\PY@bf=\textbf}
\def\PY@tok@gr{\def\PY@tc##1{\textcolor[rgb]{1.00,0.00,0.00}{##1}}}
\def\PY@tok@cm{\let\PY@it=\textit\def\PY@tc##1{\textcolor[rgb]{0.25,0.50,0.50}{##1}}}
\def\PY@tok@vg{\def\PY@tc##1{\textcolor[rgb]{0.10,0.09,0.49}{##1}}}
\def\PY@tok@m{\def\PY@tc##1{\textcolor[rgb]{0.40,0.40,0.40}{##1}}}
\def\PY@tok@mh{\def\PY@tc##1{\textcolor[rgb]{0.40,0.40,0.40}{##1}}}
\def\PY@tok@go{\def\PY@tc##1{\textcolor[rgb]{0.50,0.50,0.50}{##1}}}
\def\PY@tok@ge{\let\PY@it=\textit}
\def\PY@tok@vc{\def\PY@tc##1{\textcolor[rgb]{0.10,0.09,0.49}{##1}}}
\def\PY@tok@il{\def\PY@tc##1{\textcolor[rgb]{0.40,0.40,0.40}{##1}}}
\def\PY@tok@cs{\let\PY@it=\textit\def\PY@tc##1{\textcolor[rgb]{0.25,0.50,0.50}{##1}}}
\def\PY@tok@cp{\def\PY@tc##1{\textcolor[rgb]{0.74,0.48,0.00}{##1}}}
\def\PY@tok@gi{\def\PY@tc##1{\textcolor[rgb]{0.00,0.63,0.00}{##1}}}
\def\PY@tok@gh{\let\PY@bf=\textbf\def\PY@tc##1{\textcolor[rgb]{0.00,0.00,0.50}{##1}}}
\def\PY@tok@ni{\let\PY@bf=\textbf\def\PY@tc##1{\textcolor[rgb]{0.60,0.60,0.60}{##1}}}
\def\PY@tok@nl{\def\PY@tc##1{\textcolor[rgb]{0.63,0.63,0.00}{##1}}}
\def\PY@tok@nn{\let\PY@bf=\textbf\def\PY@tc##1{\textcolor[rgb]{0.00,0.00,1.00}{##1}}}
\def\PY@tok@no{\def\PY@tc##1{\textcolor[rgb]{0.53,0.00,0.00}{##1}}}
\def\PY@tok@na{\def\PY@tc##1{\textcolor[rgb]{0.49,0.56,0.16}{##1}}}
\def\PY@tok@nb{\def\PY@tc##1{\textcolor[rgb]{0.00,0.50,0.00}{##1}}}
\def\PY@tok@nc{\let\PY@bf=\textbf\def\PY@tc##1{\textcolor[rgb]{0.00,0.00,1.00}{##1}}}
\def\PY@tok@nd{\def\PY@tc##1{\textcolor[rgb]{0.67,0.13,1.00}{##1}}}
\def\PY@tok@ne{\let\PY@bf=\textbf\def\PY@tc##1{\textcolor[rgb]{0.82,0.25,0.23}{##1}}}
\def\PY@tok@nf{\def\PY@tc##1{\textcolor[rgb]{0.00,0.00,1.00}{##1}}}
\def\PY@tok@si{\let\PY@bf=\textbf\def\PY@tc##1{\textcolor[rgb]{0.73,0.40,0.53}{##1}}}
\def\PY@tok@s2{\def\PY@tc##1{\textcolor[rgb]{0.73,0.13,0.13}{##1}}}
\def\PY@tok@vi{\def\PY@tc##1{\textcolor[rgb]{0.10,0.09,0.49}{##1}}}
\def\PY@tok@nt{\let\PY@bf=\textbf\def\PY@tc##1{\textcolor[rgb]{0.00,0.50,0.00}{##1}}}
\def\PY@tok@nv{\def\PY@tc##1{\textcolor[rgb]{0.10,0.09,0.49}{##1}}}
\def\PY@tok@s1{\def\PY@tc##1{\textcolor[rgb]{0.73,0.13,0.13}{##1}}}
\def\PY@tok@sh{\def\PY@tc##1{\textcolor[rgb]{0.73,0.13,0.13}{##1}}}
\def\PY@tok@sc{\def\PY@tc##1{\textcolor[rgb]{0.73,0.13,0.13}{##1}}}
\def\PY@tok@sx{\def\PY@tc##1{\textcolor[rgb]{0.00,0.50,0.00}{##1}}}
\def\PY@tok@bp{\def\PY@tc##1{\textcolor[rgb]{0.00,0.50,0.00}{##1}}}
\def\PY@tok@c1{\let\PY@it=\textit\def\PY@tc##1{\textcolor[rgb]{0.25,0.50,0.50}{##1}}}
\def\PY@tok@kc{\let\PY@bf=\textbf\def\PY@tc##1{\textcolor[rgb]{0.00,0.50,0.00}{##1}}}
\def\PY@tok@c{\let\PY@it=\textit\def\PY@tc##1{\textcolor[rgb]{0.25,0.50,0.50}{##1}}}
\def\PY@tok@mf{\def\PY@tc##1{\textcolor[rgb]{0.40,0.40,0.40}{##1}}}
\def\PY@tok@err{\def\PY@bc##1{\fcolorbox[rgb]{1.00,0.00,0.00}{1,1,1}{##1}}}
\def\PY@tok@kd{\let\PY@bf=\textbf\def\PY@tc##1{\textcolor[rgb]{0.00,0.50,0.00}{##1}}}
\def\PY@tok@ss{\def\PY@tc##1{\textcolor[rgb]{0.10,0.09,0.49}{##1}}}
\def\PY@tok@sr{\def\PY@tc##1{\textcolor[rgb]{0.73,0.40,0.53}{##1}}}
\def\PY@tok@mo{\def\PY@tc##1{\textcolor[rgb]{0.40,0.40,0.40}{##1}}}
\def\PY@tok@kn{\let\PY@bf=\textbf\def\PY@tc##1{\textcolor[rgb]{0.00,0.50,0.00}{##1}}}
\def\PY@tok@mi{\def\PY@tc##1{\textcolor[rgb]{0.40,0.40,0.40}{##1}}}
\def\PY@tok@gp{\let\PY@bf=\textbf\def\PY@tc##1{\textcolor[rgb]{0.00,0.00,0.50}{##1}}}
\def\PY@tok@o{\def\PY@tc##1{\textcolor[rgb]{0.40,0.40,0.40}{##1}}}
\def\PY@tok@kr{\let\PY@bf=\textbf\def\PY@tc##1{\textcolor[rgb]{0.00,0.50,0.00}{##1}}}
\def\PY@tok@s{\def\PY@tc##1{\textcolor[rgb]{0.73,0.13,0.13}{##1}}}
\def\PY@tok@kp{\def\PY@tc##1{\textcolor[rgb]{0.00,0.50,0.00}{##1}}}
\def\PY@tok@w{\def\PY@tc##1{\textcolor[rgb]{0.73,0.73,0.73}{##1}}}
\def\PY@tok@kt{\def\PY@tc##1{\textcolor[rgb]{0.69,0.00,0.25}{##1}}}
\def\PY@tok@ow{\let\PY@bf=\textbf\def\PY@tc##1{\textcolor[rgb]{0.67,0.13,1.00}{##1}}}
\def\PY@tok@sb{\def\PY@tc##1{\textcolor[rgb]{0.73,0.13,0.13}{##1}}}
\def\PY@tok@k{\let\PY@bf=\textbf\def\PY@tc##1{\textcolor[rgb]{0.00,0.50,0.00}{##1}}}
\def\PY@tok@se{\let\PY@bf=\textbf\def\PY@tc##1{\textcolor[rgb]{0.73,0.40,0.13}{##1}}}
\def\PY@tok@sd{\let\PY@it=\textit\def\PY@tc##1{\textcolor[rgb]{0.73,0.13,0.13}{##1}}}

\def\PYZbs{\char`\\}
\def\PYZus{\char`\_}
\def\PYZob{\char`\{}
\def\PYZcb{\char`\}}
\def\PYZca{\char`\^}
\def\PYZsh{\char`\#}
\def\PYZpc{\char`\%}
\def\PYZdl{\char`\$}
\def\PYZti{\char`\~}
% for compatibility with earlier versions
\def\PYZat{@}
\def\PYZlb{[}
\def\PYZrb{]}
\makeatother
% pygments rész vége

\newcommand{\figs}{{cn_edu/pdf}}
\newcommand{\deb}{{\em deb}}
\newcommand{\apt}{{\tt apt}}
\newcommand{\kin}{$k_{\rm in}$}
\newcommand{\kout}{$k_{\rm out}$}
\newcommand{\megoldasjon}{\vfill
\begin{tikzpicture}
    \node[draw,thick,green!50!black,circle,inner sep=1] (A) at (0,0) {M};
\end{tikzpicture}
{\color{red}Megoldás jön!}
Lapozás előtt próbálja megoldani a feladatot!}

\newcommand{\modul}[1]{\section{modul -- #1}}
\newcommand{\modulcel}{\par \textbf{A modul célja:} }
\newenvironment{feladat}{\par \textbf{Feladat:}}{}
\newenvironment{kovetelmenyek}{
    \begin{itemize}\itemsep0pt\parskip0pt}
    {\end{itemize}}

\newcommand{\kilencven}{\textbf{\color{red}90}}

\author{Horváth Árpád <horvath.arpad@amk.uni-obuda.hu>}
\subtitle{\"Osszetett h\'al\'ozatok vizsg\'alata}
\institute[ÓE AMK]{Óbudai Egyetem\\
  Alba Regia Műszaki Kar (AMK)\\
  Székesfehérvár}

% Delete this, if you do not want the table of contents to pop up at
% the beginning of each subsection:
\AtBeginSection[]
{
  \begin{frame}<beamer>
    \frametitle{Vázlat}
    \tableofcontents[currentsection]
  \end{frame}
}

\AtBeginSubsection[]
{
  \begin{frame}<beamer>
    \frametitle{Vázlat}
    \tableofcontents[currentsection,currentsubsection]
  \end{frame}
}

\setbeamertemplate{navigation symbols}{}

\newlength{\maxheight}
\setlength{\maxheight}{0.63\textwidth}




\title{Regul\'aris kifejez\'esek}
\subtitle{}

\author{Horváth Árpád <horvath.arpad.szfvar@gmail.com>}
\institute[]{}

\begin{document}


\frame{\maketitle}

\begin{frame}<beamer>
  \frametitle{Vázlat}
  \tableofcontents
\end{frame}

\begin{frame}[fragile]
    {Egy nyelv}{Melyik lehet? Melyikek hibásak? Hogyan ellenőriznénk programmal?}

\begin{verbatim}
ABC-001
TRA-548
AAA-000
-01
AAA-
NEKEM-8
KISAPA-01
FERI-12
KISAPA-0122333
\end{verbatim}
\end{frame}

\begin{frame}
    \noindent
    \includegraphics[width=.48\linewidth]{rendszamok/nekem-8_reszlet.jpg} 
    \includegraphics[width=.48\linewidth]{rendszamok/qqriq-7.jpg} 

    \noindent
    \includegraphics[width=.48\linewidth]{rendszamok/asso-99.png} 
    \includegraphics[width=.48\linewidth]{rendszamok/siker-1.jpg} 
\end{frame}

\begin{frame}
    \noindent
    \includegraphics[width=.98\linewidth]{rendszamok/jozsi-4.jpg} 
\end{frame}

\begin{frame}
    \noindent
    \includegraphics[width=.98\linewidth]{rendszamok/Ewing_1.jpg}
\end{frame}

\begin{frame}[fragile]
    {Egy nyelv}

Benne van:
\begin{verbatim}
ABC-001
AAA-000  talán, nem kukacoskodunk :-)
TRA-548
FERI-12
NEKEM-8
EWING-1
\end{verbatim}

Nincs benne:
\begin{verbatim}
-01
AAA-
KISAPA-01
KISAPA-0122333
\end{verbatim}
\end{frame}

\begin{frame}[fragile]
    {Másik nyelv}{Melyik lehet? Melyikek hibásak? Hogyan ellenőriznénk programmal?}

\begin{verbatim}
joci@csillagasz.at
horvath.arpad@amk.uni-obuda.hu
x_ipszilon@gmail.com
very.common@example.gov
bobebaba13@futrinka.mtv.hu
bobe@baba13@futrinka.mtv.hu
abc.example.com
bobebaba13@futrinka.mtv.
bobebaba13@futrinka..hu
\end{verbatim}
\end{frame}

% \begin{frame}[fragile]
%     {Másik nyelv}{Melyik lehet? Melyikek hibásak? Hogyan ellenőriznénk programmal?}
% 
% \begin{verbatim}
% 192.168.3.26
% 1.5.222.54
% 192.168.3.026
% 1992.168.3.26
% 184..233.45
% 011.12.022.03
% 560.370.13.25
% \end{verbatim}
% \end{frame}

\section{Hol alkalmazhatunk reguláris kifejezéseket}

\begin{frame}
    {Mire jók a reguláris kifejezések}

    A reguláris kifejezések a reguláris nyelvek tömör leírását adják. A
    reguláris kifejezések egy algoritmussal véges automatává alakítható,
    ami alapján készíthető olyan program vagy hardvereszköz, amely
    ellenőrizni tud egy reguláris kifejezést. Ezzel az algoritmussal nem
    foglalkozunk, (az megtalálható a Bach-könyvben, illetve az átalakítás
    gyakorlolható a JFLAP programmal).

    \vfill
    Mi a reguláris kifejezések létrehozásával fogunk foglalkozni. Az
    ellenőrzést más tárgyakban rábízzuk a programnyelvek kész eszközeire.

    \vfill
    A programnyelvek és az XML/HTML fájlok környezetfüggetlen nyelvének
    szintaktikai elemzése kívül esik a reguláris kifejezések hatókörén.
\end{frame}

\begin{frame}
    {Reguláris nyelvek, kifejezések és nyelvtanok ill. véges automaták
    kapcsolata}

    \begin{tikzpicture}[>=latex]
        \node[box,fill=red!20] (nyelv) at (0,0) {reguláris nyelv};
        \node[box] (kif) at (0,2) {reguláris kifejezés};
        \node[box] (nyelvtan) at (210:3.5) {reguláris nyelvtan};
        \node[box] (automata) at (330:3.5) {véges automata};
        \draw[->] (kif) -- (nyelv);
        \draw[->] (nyelvtan) -- (nyelv);
        \draw[->] (automata) -- (nyelv);
        \draw[<->] (automata) -- (nyelvtan);
        \draw[->,very thick] (kif) edge [bend left] (automata);
    \end{tikzpicture}
\end{frame}

\begin{frame}
    {Reguláris nyelvek, kifejezések és nyelvtanok ill. véges automaták
    kapcsolata}

    \begin{tikzpicture}[>=latex]
        \node[box,fill=red!20] (nyelv) at (0,0) {reguláris nyelv\\
        \small szavak
        halmaza};
        \node[box] (kif) at (0,2) {reguláris kifejezés\\ \small tömör leírás};
        \node[box] (nyelvtan) at (210:3.5) {reguláris
        nyelvtan\\ \small levezetési szabályok};
        \node[box] (automata) at (330:3.5) {véges automata\\ \small állapotgráf};
        \draw[->] (kif) -- (nyelv);
        \draw[->] (nyelvtan) -- (nyelv);
        \draw[->] (automata) -- (nyelv);
        \draw[<->] (automata) -- (nyelvtan);
        \draw[->,very thick] (kif) edge [bend left] (automata);
    \end{tikzpicture}
\end{frame}

\begin{frame}[fragile]
    {Hol találkozhatunk reguláris kifejezésekkel?}{Programnyelvek}

    Van ahol a programnyelv része:

    \begin{itemize}
        \item perl, awk
        \item JavaScript
        \item PostgreSQL
    \end{itemize}

    \vfill
    vagy standard könyvtárában benne van:
    \begin{itemize}
        \item C\#
        \item Python
    \end{itemize}
\end{frame}

\begin{frame}[fragile]
    {És még hol találkozunk regexp-el?}

    (regular expression = regexp)

    \vfill
    Ezekben kereshetünk /-rel: vim man less Firefox

    \vfill
    Parancsok, amelyekben használható: grep sed

    \vfill
    A \alert{django} web-keretrendszer a kapott URL-eket reguláris
    kifejezésekkel hasonlítja össze, így dől el, mit csinál, milyen
    paraméterekkel.
\end{frame}

\section{A reguláris kifejezések minimális nyelve}

\begin{frame}
    Az alábbiakban r, r1, r2 egy-egy érvényes reguláris kifejezést jelent.

    A reguláris kifejezések leírhatóak a következő pár szabály
    alkalmazásával:

    \vfill
    \begin{tabular}{rl}
        r* & r-ből 0-ra vagy többre illeszkedik\\
        r1|r2 & vagy r1-re, vagy r2-re illeszkedik\\
        (r) & csoportosítás\\
        r1r2 & összefűzés\\
    \end{tabular}

    \vfill
    Példák:\\
    \begin{tabular}{rll}
        kifejezés & illeszkedik erre & nem illeszkedik erre\\
        \hline
        ab*a & aa aba abbba & ab*a\\
        (ab)*a & a aba ababa ababababa & ab*a abab\\
        (a|A)lma & Alma alma & \\
        colo(|u)r & color colour& \\
    \end{tabular}
\end{frame}

\begin{frame}
    {Feladat}

    Milyen reguláris kifejezés illik az alábbi szavakra?

    \begin{itemize}
        \item ea ema emma emmma \ldots emmmmmmmma \ldots
    \end{itemize}

    A feladatokat követő dián megoldás következik.
\end{frame}

\begin{frame}
    {Megoldás}

    Milyen reguláris kifejezés illik az alábbi szavakra?

    \begin{itemize}
        \item ea ema emma emmma \ldots emmmmmmmma \ldots
	\item Legkézenfekvőbb megoldás: em*a
    \end{itemize}
\end{frame}

\begin{frame}
    {Feladat}

    Milyen reguláris kifejezés illik az alábbi szavakra?

    \begin{itemize}
        \item ema emma emmma \ldots emmmmmmmma \ldots,\\
            de ea nem
    \end{itemize}

    A feladatokat követő dián megoldás következik.
\end{frame}

\begin{frame}
    {Megoldás}

    \begin{itemize}
        \item ema emma emmma \ldots emmmmmmmma \ldots,\\
            de ea nem\\
	\item Legkézenfekvőbb megoldás: emm*a
    \end{itemize}
\end{frame}

\begin{frame}
    {Feladat}

    Milyen reguláris kifejezés illik az alábbi szavakra?

    \begin{itemize}
        \item alma1 alma2 alma3 emma1 emma2 emma3
    \end{itemize}

    A feladatokat követő dián megoldás következik.
\end{frame}

\begin{frame}
    {Megoldás}

    \begin{itemize}
        \item alma1 alma2 alma3 emma1 emma2 emma3\\
	\item Két viszonylag kézenfekvő megoldás is van:\\
            (al|em)ma(1|2|3)\qquad   vagy\\
            (alma|emma)(1|2|3)
    \end{itemize}
\end{frame}

\begin{frame}
    {Jelölésmódok}

    A Bach Iván könyvben a választás jelölése +. Itt a
    reguláris kifejezések esetén  gyakorlatban is használt | jelölést
    használjuk. A többi jelölés egyezik az itteniekkel.

    A reguláris kifejezések gyakorlatában a + jel a *-hoz hasonló, de 1
    vagy több előfordulást jelöl.
\end{frame}

\begin{frame}[fragile]
    {Speciális jelentés levédése}

    A * és \verb"|" jeleknek speciális jeletése van, de néha szükséges, hogy *
    vagy \verb"|" jelekre illesszünk. Ilyenkor általában a \verb"\" karaktert
    illesztjük a speciális jeletésű karakter elé, hasonlóan, mint ahogy
    az újsort jelöljük a karakterláncok esetén (\verb!\n!).

    Az egyes alkalmazások reguláris nyelvei eltérnek abban, hogy a
    levédett vagy a levédetlen alak jelenti a speciális jelentést. Ezek
    között az egyik véglet a Python nyelv re modulja, ahol általában a
    speciális jelentést nem kell levédeni, a másik véglet a Vim és a
    grep, ahol általában a speciális jelentést kell levédeni.
\end{frame}

\begin{frame}[fragile]
    {A speciális jelentések levédése (folyt.)}


    A * és a . (pont) jelöléseket egyik esetben sem kell levédeni a
    speciális jelentéshez (és más általam ismert programnyelvben sem).
    (A .  tetszőleges karakterre illeszkedik.)  A * és . karakterekre a
    \verb"\*" és \verb"\." kifejezés illeszkedik. A \verb"\" (visszaper,
    backslash) karakterre a \verb"\\" reguláris kifejezés illeszkedik.

    Az további példákban a Python jelölésmódját mutatom be, a
    példamegoldáshoz azt kell tudni, és aszerint kell megoldani a
    feladatokat.

    A Python nyelvben a csoportosítás zárójelét nem kell levédeni, a
    \verb"\(" és a \verb"\)" illeszkedik a zárójelekre. Szintén nem kell
    levédeni a választás függőleges vonalát, tehát a \verb"\|"
    a \verb"|" karakterre illeszkedik. A jelek, amelyeket le
    kell védeni, mert speciális jelentésük van \verb"*.|\[]{}+?^$"
\end{frame}

\begin{frame}[fragile]
    {Feladat}
    Hogyan illesztenénk olyan kifejezésre, amely 0 vagy több a-betű után
    egy *, majd egy vagy több b betű követ?

    \megoldasjon
\end{frame}

\begin{frame}[fragile]
    {Megoldás}
    Hogyan illesztenénk olyan kifejezésre, amely 0 vagy több a-betű után
    egy *, majd egy vagy több b betű követ?\\
    \verb!a*\*bb*! \quad vagy \quad \verb!a*\*b+!\\
    utóbbi esetben eltérő lehet, hogy a + elé kell-e \verb|\|.
\end{frame}

\begin{frame}[fragile]
    {Feladat}
    Hogyan illeszthetünk olyan kifejezést, amelynél két bináris számjegy
    után . (pont) majd egy bináris számjegy következik?

    Pl: 01.1  10.0 11.1

    \megoldasjon
\end{frame}

\begin{frame}[fragile]
    {Megoldás}

    Hogyan illeszthetünk olyan kifejezést, amelynél két bináris számjegy
    után . (pont) majd egy bináris számjegy következik?\\
    \verb!(0|1)(0|1)\.(0|1)!
\end{frame}

\begin{frame}
    [fragile]{Feladat}

    Melyek illeszkednek az első reguláris kifejezésekre az azt követő
    szavak közül?

\begin{Verbatim}[commandchars=!\{\}]
!alert{a.*b}  accb a..b a.*b a**b a...b a*b* A.b
!alert{a\.*b} a...b a\.b a\..b a\ccb ab a.b
!alert{a(ab)*b} ab a(ab)))b aab aababb ababab aabababb
!alert{(0|1|2)*(a|b)*} 0|1|2a|b ab12 (21)(ba)
              012ab 210ba ab012 012*ab*
!alert{(|o|O)buda(|i)} buda budai |budai Obuda Obudai
            (Obudai) (o)buda(i) obuda )buda(i)
!alert{\((0|1|2),(2|3)\)} 2,3  \(2,2\) (0,1) (0,3 (1,3)
            ((1),(2)) (13) (2,3) (1,2)
\end{Verbatim}
\end{frame}

\begin{frame}[fragile]
    {Megoldások}

    Az egyes sorokból a következőek helyesek csak:

\begin{Verbatim}[commandchars=!\{\}]
!alert{a.*b}  accb a..b a.*b a**b a...b
!alert{a\.*b} a...b ab a.b
!alert{a(ab)*b} ab aababb aabababb
!alert{(0|1|2)*(a|b)*} 012ab 210ba
!alert{(|o|O)buda(|i)} buda budai Obuda Obudai obuda
!alert{\((0|1|2),(2|3)\)} (1,3) (2,3) (1,2)
\end{Verbatim}

    Zárthelyiben az előzőhöz hasonló kiválasztásos feladatok esetén a
    helytelenül aláhúzott változatok pontlevonással járnak.
\end{frame}

\section{A reguláris kifejezések bővített nyelve}

\utolsooldal


\begin{frame}[fragile]
    {Miért kell bővíteni?}

    A reguláris kifejezések leírásához elegendő a kezdetben említett
    négy lehetőség, mégis sok esetben elég körülményes azokkal felírni
    azt, amit illeszteni szeretnénk. Az egyik nehézség, amikor \alert{nagy
    számú karakter valamelyikére}, például tetszőleges betűre, kell
    illeszkedni, az elég hosszan írható csak le.  Ezért bevezetünk egy
    \alert{betű- vagy számintervallumok megadására is alkalmas
    halmazjelölést}.

    \vfill
    Egy másik probléma, hogy az \alert{ismétlések számát} nem olyan
    egyszerű megadni. Hogyan adhatjuk meg azt, hogy a b betű egymás
    után legalább 2-szer, de maximum 7-szer szerepelhet? Például így:\\
    \verb"(bb|bbb|bbbb|bbbbb|bbbbbb|bbbbbbb)" \quad vagy így\\
    \verb"bb(|b|bb|bbb|bbbb|bbbbb)" \quad egyik sem kellemes.
\end{frame}

\begin{frame}[fragile]
    {Halmazjelölés}

    A szögletes zárójel halmaz magadására alkalmas. Legegyszerűbb
    formájában a felsorolt karakterek halmazát jelenti:

    \verb+[ilab@._]*+ illeszkedik az \verb+ali_baba@abba.lb+ szóra.

    (A már említett . (pont) karaktert itt nem kell levédeni, nincs
    értelme a speciális jelentésnek.)
    
    \vfill
    Megadhatunk szám és betűintervallumokat is, amennyiben a kezdő és
    végkarakter közé $-$ jelet (kötőjel) rakunk.

    Például a decimális számjegyeket, a(z ékezet nélküli) nagybetűket,
    illetve a kisbetűket a következőképpen adhatjuk meg:\\
    \verb"[0-9]" \quad
    \verb"[A-Z]" \quad
    \verb"[a-z]"
\end{frame}

\begin{frame}[fragile]
    {Feladat}

    Vajon hogyan adható meg egy nyolcas számrendszerbeli (oktális)
    számjegy az intervallumjelöléssel?

    \coloruncover{1}{[0-7]}

    \uncover<2->{Hogyan adható meg egy hexadecimális számjegy? (Egyszerűség
    kedvéért a betűk nagybetűvel szerepeljenek: A, B, C, D, E, F!)}

    \coloruncover{2}{([0-9]|[A-F])}

    \uncover<3->{ Az intervallumok összevonhatóak egy szögletes zárójelbe
    szorosan egymás mellé írva: [0-9A-F] }
\end{frame}

\begin{frame}[fragile]
    {Karakterhalmazok megadása}

    Az előbbi jelölés tetszőleges karakterhalmazok megadására is alkalmas.
    Az [c0-3@] kifejezés a c, 0, 1, 2, 3 és @ mindegyikére illeszkedik.

    Ha egy halmaznak $-$ jelre (kötőjel) is illeszkednie kell, akkor
    előre rakjuk a $-$ jelet, vagy védjük le.

    \verb![-+03]! \qquad $-$, +, 0 és 3 karakterekre illeszkedik

    \verb![+0\-3]! \qquad ez is

    \verb![+0-3]! \qquad +, 0, 1, 2 és 3 karakterekre illeszkedik
\end{frame}

\begin{frame}[fragile]
    {Feladat}

    Egy intézetben két épület van, amelyeket K és F betűvel jelölnek. A
    termek sorszámozása mindegyik épületben 0-tól 99-ig történik. A
    termeket a következőképp adják meg K.00 K.01 \ldots K.99, F.00,
    F.01, \ldots F.99. Írjon reguláris kifejezést, amely mindegyik
    teremnévre illeszkedik!

    \megoldasjon
\end{frame}

\begin{frame}[fragile]
    {Megoldás}

    \verb![FK]\.[0-9][0-9]!

    Halmazjelölés nélkül:

    \verb!(F|K)\.(0|1|2|3|4|5|6|7|8|9)(0|1|2|3|4|5|6|7|8|9)!
\end{frame}

\begin{frame}[fragile]
    {Speciális jelölések karakterhalmazokra}

    Néhány speciális halmaz jele látható az alábbi halmazban. Elég az
    első hármat tudni.

    \vfill
    \begin{tabular}{rl}
        . (pont)  & tetszőleges karakterre\\
        \verb"\d" & decimális karakterek [0-9]\\
        \verb"\w" & \verb"[0-9a-zA-Z_]"\\
        \verb"\W" & az előbbi ellentettje\\
        \verb"\s" & \verb"[ \t\n\r\f\v]"\\
        \verb"\S" & az előbbi ellentettje\\
    \end{tabular}

    \vfill
    A \verb"\w" tehát a változónevekben szereplő karaktereket
    tartalmazza a \verb"\s" pedig a szóközt, a tabulátort és az újsort
    is magában foglaló whitespace karaktereket.

    \vfill
    Ezek halmazon belül is használhatóak, illetve a halmazjelölés
    szögletes zárójelén belül a pont elveszti speciális jelentését.
    Emiatt a \verb"[-.\w]+" illeszkedik az
    \verb"Django_10.My-Server.hu" kifejezésre.
\end{frame}

\begin{frame}[fragile]
    {Ismétlések}

    Az alábbi ismétlést jelentő jeleket használhatjuk. $n$ és $m$
    mindenhol természetes számot jelent.

    \vfill
    \begin{tabular}{rl}
        * & 0 vagy több ismétlődés\\
        + & 1 vagy több ismétlődés\\
        ? & 0 vagy 1 ismétlődés (van vagy nincs)\\
        \verb"{n}" & pontosan n ismétlődés\\
        \verb"{n,m}" & min. n, max. m ismétlődés\\
        \verb"{,m}" & max. m ismétlődés\\
        \verb"{n,}" & min. n ismétlődés
    \end{tabular}
\end{frame}

\begin{frame}
    [fragile]{Feladat}

    Melyek illeszkednek az első reguláris kifejezésekre az azt követő
    szavak közül?

\begin{Verbatim}[commandchars=!()]
!alert([A-Z]{2}\d{5}) CA12345 12345CA C1B2345 XX111 AA54321
\end{Verbatim}
\begin{Verbatim}[commandchars=ABC]
AalertB(1?\d{2}|2[0-4]\d|25[0-5])C 256 1222 255 00 1?44 244 133
AalertB(da){2,}dogC dadog da{2,}dog dadadadog dadadog
\end{Verbatim}
\begin{Verbatim}[commandchars=!\{\}]
!alert{[.\w]+}  ... al-Hakim man_page re_howto.pdf index.html
!alert{b[a-j]+} baba baj b bujj b[a] b[a]+ ba+
!alert{https?} https? http https httpst httpsa
!alert{\d+\.\d*[eE]-?\d+} 6.02e23 6e23 1.38E-23 1.2E- .5e2
      1.E-6 2.333[eE]-?5
!alert{\(\d+\+\d+\)} (33\+44) (5+6) (4-5) (33++44) (+22) (123+456)
\end{Verbatim}
\end{frame}

\begin{frame}
    [fragile]{Megoldás}

    Csak ezek:

\begin{Verbatim}[commandchars=!()]
!alert([A-Z]{2}\d{5}) CA12345 AA54321
\end{Verbatim}
\begin{Verbatim}[commandchars=ABC]
AalertB(1?\d{2}|2[0-4]\d|25[0-5])C 255 00 244 133
AalertB(da){2,}dogC dadadadog dadadog
\end{Verbatim}
\begin{Verbatim}[commandchars=!\{\}]
!alert{[.\w]+}  ... man_page re_howto.pdf index.html
!alert{b[a-j]+} baba baj
!alert{https?} http https
!alert{\d+\.\d*[eE]-?\d+} 6.02e23 1.38E-23 1.E-6
!alert{\(\d+\+\d+\)} (5+6) (123+456)
\end{Verbatim}

A második kis hibával a 0 és 255 közötti számokat adja. A kis hiba, hogy
a 00 01 \ldots 09 formájú számok is vannak közöttük. Kis munkával
javítható: \verb!([0-9]|[1-9]\d|1\d{2}|2[0-4]\d|25[0-5])!
\end{frame}

\begin{frame}
    {Feladat}

    Hogyan illeszthetünk a szokásos magyar rendszámokra?

    \vfill
    Az alábbiak arra jók, hogy lássuk mire lehet/érdemes reguláris
    kifejezést írni. Ennyire összetett feladat nem lesz számonkéréskor.

    \vfill
    Próbáljunk egy közelítő reguláris kifejezést készíteni a
    következőkre:
    \begin{itemize}
        \item IP-címek
        \item e-mailek
        \item URL-ek
    \end{itemize}

    \megoldasjon
\end{frame}

\begin{frame}[fragile]
    {Megoldások (nem mind)}

    \verb"[A-Z]{3}-[0-9]{3}"

\vfill
IP-k (több sorban, re.VERBOSE mód)
\begin{verbatim}
(
 (1?\d\d|2[0-4]\d|25[0-5])  \.
){3}
(1?\d\d|2[0-4]\d|25[0-5])
\end{verbatim}

\vfill
e-mailek (szóközökkel elválasztva, re.VERBOSE mód)
\begin{verbatim}
 [-\w.]+   @   ([-\w]+\.)+   (hu|de|uk|...|com|org|...)
\end{verbatim}
\end{frame}

\begin{frame}<beamer>
    {Zárójelpárosítás}

    Lehet-e zárójelpárosításokat ellenőrizni reguláris kifejezésekkel?

    \begin{itemize}[<+->]
        \item Korlátozott számú kapcsos zárójelet hogyan ellenőrizhetek?
        \item Korlátlan számú kapcsos zárójelet hogyan ellenőrizhetek?
        \item Korlátozott számú vegyes zárójelet hogyan ellenőrizhetek?
    \end{itemize}
\end{frame}

\begin{frame}<beamer>
    [fragile]
    {Zárójelpárosítás}

    Lehet-e zárójelpárosításokat ellenőrizni reguláris kifejezésekkel?

    \begin{itemize}[<+->]
        \item Korlátozott számú kapcsos zárójelpár:
            Pl. max 2 darab egymásba ágyazott, azaz \verb"{{}} vagy {}"
            vagy $\varepsilon$.\\
            \verb!({({})?})?!   \\
            (átláthatóság kedvéért a kapcsos zárójelek levédését
            elhagytam).\\
            Háromnál sorrendek \verb"{}{}{} {{}{}} {{{}}}"
        \item Korlátlan számú kapcsos zárójelpár:
            A zárójelezések nyelve környezetfüggetlen nyelv, nem fér
            bele a regulárisba $\Rightarrow$ nincs hozzá reguláris
            kifejezés. \Smiley{-0.51}
        \item Korlátozott számú vegyes zárójelpár:
            Nagyon bonyolult. Ismerni kell a lehetséges sorrendeket.
        \item Veremautomatával egyszerű lesz mindegyik. \Smiley{1}
    \end{itemize}
\end{frame}




\begin{frame}
    {Az elméleti rész összefoglalója}

    \begin{itemize}[<+->]
        \item A reguláris kifejezések tömör leírását adják a reguláris nyelveknek.
        \item Alkalmasak
            \begin{itemize}
                \item számunkra érdekes adatok megkeresésére (grep),
                    cseréjére (sed, vim),
                \item  hibás alakú adatok kiszűrésére (validálás).
            \end{itemize}
        \item Tömör jelölések karakterhalmazokra és
            ismétlésre. \Smiley{.5}
        \item Reguláris nyelveket adnak meg $\Rightarrow$ programok
            szintaktikai elemzésre nem alkalmasak. \Smiley{-.8}
        \item Zárójelpárosítás ellenőrzésére korlátozottan alkalmas. \Smiley{-.2}
    \end{itemize}
\end{frame}


\begin{frame}
    {Mikkel foglalkoztunk eddig?}{Összetevők és kapcsolatok}

    \uncover<1-5>{Először az ,,összetevők''.}
    \uncover<6-8>{Mindhárom \emph{egyértelműen} meghatározza a nyelvet. Milyen \alert{igéket} használunk az egyes esetekben?}
    %\definecolor{uncovercolor}{rgb}{0.8,0.8,1}
    \begin{tikzpicture}[>=latex]
	\node[box,fill=blue!20] (nyelv) at (0,0) {\coloruncover{2}{reguláris nyelv}};
	\node[box] (kif) at (0,2) {\coloruncover{1}{reguláris} kifejezés};
	\node[box] (nyelvtan) at (210:3.5) {\coloruncover{3}{reguláris nyelvtan}};
        \node[box] (automata) at (330:3.5) {\coloruncover{5}{det.}\coloruncover{4}{véges automata}};
	\draw[->] (kif) -- node[left]{\coloruncover{8}{illeszkedik a szavaira}} (nyelv);
        \draw[->] (nyelvtan) -- node[left]{\coloruncover{6}{generálja}} (nyelv);
        \draw[->] (automata) -- node[left]{\coloruncover{7}{felismeri}}(nyelv);
        \draw[<->] (automata) -- (nyelvtan);
        \draw[->,very thick] (kif) edge [bend left] (automata);
    \end{tikzpicture}

    \uncover<9>{Nézzünk pár feladatot a véges automaták és reguláris kifejezések kapcsolatára.}
\end{frame}

\section{Automatás feladatok}
\begin{frame}
    {Feladat}

    Milyen reguláris kifejezéssel írható le az alábbi véges automata által
    felismert nyelv? Írjuk le bővítetlen reguláris kifejezéssel és tömörebben
    a bővített reguláris kifejezés használatával!


\begin{tikzpicture}[scale=2]
\node[state] (S) at (0,0) {$q_0$} edge[<-,thick] (-.5,0);
\node[state] (A) at (1,0) {$q_1$};
\node[finalstate] (B) at (2,0) {$q_2$};
\path[thick,->] (S) edge node[above] {a} (A)
                     edge[loop above] node[near end,right] {b} ()
                (A) edge[loop above] node[near end,right] {a} ()
                     edge node[above] {b} (B)
                (B) edge[loop above] node[near end,right] {a}  ()
                     edge[loop below] node[near end, left] {b}  ();
\end{tikzpicture}

   Írjuk le a nyelvet halmazjelöléssel is!

   \megoldasjon
\end{frame}

\begin{frame}
    {Megoldás}

    b*aa*b(a|b)* \quad vagy \quad b*a+b[ab]*

    Az első megoldás nem használja a bővített nyelvet.

    \[{\cal L} = \{b^ia^jbw \mid i\ge 0, j> 0, w \in \{a,b\}^*\}\]
\end{frame}

\begin{frame}
    {Feladat}

    Milyen reguláris kifejezéssel írható le az alábbi véges automata által
    felismert nyelv? Használhatja a bővített reguláris kifejezést!

\begin{tikzpicture}[scale=2]
\node[finalstate] (S) at (0,0) {$q_0$} edge[<-,thick] (-.5,0);
\node[state] (A) at (1,0) {$q_1$};
\node[finalstate] (C) at (2,0) {$q_2$};
\node[finalstate] (B) at (1,-1) {$q_3$};
\path[thick,->] (S) edge node[above] {a} (A)
                (A) edge[loop above] node[near end,right] {b} ()
                     edge node[above] {c} (C)
                     edge node[right] {b} (B)
                (C) edge[loop above] node[near end,right] {c}  ();
\end{tikzpicture}

   Írjuk le a nyelvet halmazjelöléssel is!

   (Létrehozhatja a nyelvtant is, ami ugyanazt a nyelvet generálja,
   amit az automata felismer.)

   \megoldasjon
\end{frame}

\begin{frame}
    {Megoldás}

    (ab+|ab*c+|) \quad esetleg \quad (ab*(b|c+)|)

    Az első változat három függőleges vonallal elválasztott tagból áll:
    az első tag, amikor a B-be érek, a második, amikor a C-be érek, a
    harmadik amikor S-ben maradok üres szóval. Ugyanebben a sorrendben
    szerepelnek az alábbi halmazok, amelyeknek az unióját vesszük.


    \[{\cal L} = \{ab^i \mid i>0\} \cup \{ab^ic^j \mid i \ge 0, j>0\}
    \cup \{\varepsilon\} \]
\end{frame}

\begin{frame}
    {Feladat}

    Rajzolja le olyan véges automaták állapotgráfjait, amely az alábbi
    reguláris kifejezések által megadott nyelvet ismeri fel!

    \begin{itemize}
        \item ab*c
        \item 0(12*|1+)
        \item a(b*c|a+)  ez nehezebb, ilyen nehéz nem lesz.
    \end{itemize}

    \megoldasjon
\end{frame}


\begin{frame}
    {Megoldás}

    ab*c legkézenfekvőbb megoldása:

\begin{tikzpicture}[scale=2]
\node[state] (S) at (0,0) {$q_0$} edge[<-,thick] (-.5,0);
\node[state] (A) at (1,0) {$q_1$};
\node[finalstate] (B) at (2,0) {$q_2$};
\path[thick,->] (S) edge node[above] {a} (A)
                (A)
                     edge[loop above] node[right,near end] {b} ()
                     edge node[above] {c} (B)
                ;
\end{tikzpicture}

\vfill
    0(12*|1+) egy lehetséges megoldása:

\begin{tikzpicture}[scale=2]
\node[state] (S) at (0,0) {$q_0$} edge[<-,thick] (-.5,0);
\node[state] (A) at (1,0) {$q_1$};
\node[finalstate] (C) at (2,0) {$q_2$};
\node[finalstate] (B) at (1,-1) {$q_3$};
\path[thick,->] (S) edge node[above] {0} (A)
                (A)
                     edge node[above] {1} (C)
                     edge node[right] {1} (B)
                (C) edge[loop below] node[near end,left] {1}  ()
                (B) edge[loop right] node[right] {2}  ();
\end{tikzpicture}
\end{frame}

\begin{frame}
    {Reguláris kifejezésből véges automata}

    A reguláris kifejezések illesztéséhez nem ötletszerű megoldások, hanem
    algoritmusok szükségesek. Az, hogy bármely reguláris kifejezés esetén meg
    tudjuk alkotni a véges automatát, amellyel meg tudjuk vizsgálni az egyes
    szavakat. \alert{Van erre algoritmus}, de mi nem tanuljuk. (A Bach-könyvben
    vagy máshol megnézheti, aki akarja, és a JFLAP-pel gyakorolhatja, de
    számunkra nem szükséges.)
\end{frame}


\section{Gyakorlati tudnivalók}

\begin{frame}
    {Mihez szükséges?}

    A további gyakorlati tudnivalóak nem szükségesek a Formális nyelvek
    feladatainak a megoldásához, hasznosak viszont a Linux alkalmazása
    kurzushoz, mivel a grep és a django használatakor (URL-értelmezés,
    bevitt adatok ellenőrzése=validálása) is szükséges dolgok vannak
    hátra. Hasznos továbbá, ha valamikor a Pythonban saját programot
    szeretnénk írni, ami reguláris kifejezéseket használ.
\end{frame}

\begin{frame}[fragile]
    {A sor eleje és vége}

    A sor elejére a \verb"^" (kalap) a végére a \verb"$" karakterek
    illeszkednek.

    \vfill
    Ha például a re.search függvénnyel keresünk, akkor az "aba"
    kifejezés illeszkedni fog az "Alabama" karakterláncra, annak aba
    részére. A \verb'"^aba"' nem fog illeszkedni rá, de az "abama" és
    "abakusz" karakterláncra igen. A \verb'"^aba$"' egyedül az "aba"
    karakterláncra fog illeszkedni.

    \vfill
    A Formális nyelvek rész feladatai teljes illeszkedést feltételeztek,
    mintha minden kifejezés elején ott lenne a \verb"^" és a végén a
    \verb"$". Ez a megszokott mód az elméleti tárgyalások során, például
    a Bach Iván könyvében is, ha például egy nyelvet reguláris
    kifejezéssel írunk le.
\end{frame}

\begin{frame}[fragile]
    {A nyers karakterláncok és a bő írásmód}


    A Python nyelv \alert{re} modulja képes reguláris kifejezéseket kezelni.

    \vfill
    Ha \alert{r} betűt írunk a karakterlánc elé, a \verb|\| nem
    viselkedik speciális karakterként $\Rightarrow$ könnyebben írhatunk
    reguláris kifejezéseket.

    \vfill
    A re.VERBOSE paraméterrel sorokra bonthatjuk, és megjegyzésekkel
    láthatjuk el a reguláris kifejezéseket.
\end{frame}

\begin{frame}
    [fragile]{Példa az r"" és a re.VERBOSE használatára}


\begin{Verbatim}[commandchars=\\\{\}]
\PY{c}{\PYZsh{} r nélkül}
\PY{n}{re}\PY{o}{.}\PY{n}{search}\PY{p}{(}\PY{l+s}{"}\PY{l+s+se}{\PYZbs{}\PYZbs{}}\PY{l+s}{d+}\PY{l+s+se}{\PYZbs{}\PYZbs{}}\PY{l+s}{.}\PY{l+s+se}{\PYZbs{}\PYZbs{}}\PY{l+s}{d*}\PY{l+s}{"}\PY{p}{,} \PY{l+s}{"}\PY{l+s}{123.6}\PY{l+s}{"}\PY{p}{)}

\PY{c}{\PYZsh{} r-rel}
\PY{n}{re}\PY{o}{.}\PY{n}{search}\PY{p}{(}\PY{l+s}{r"}\PY{l+s}{\PYZbs{}}\PY{l+s}{d+}\PY{l+s}{\PYZbs{}}\PY{l+s}{.}\PY{l+s}{\PYZbs{}}\PY{l+s}{d*}\PY{l+s}{"}\PY{p}{,} \PY{l+s}{"}\PY{l+s}{123.6}\PY{l+s}{"}\PY{p}{)}

\PY{c}{\PYZsh{} r-rel és re.VERBOSE-zel}
\PY{n}{re}\PY{o}{.}\PY{n}{search}\PY{p}{(}\PY{l+s}{r"""}
\PY{l+s}{   }\PY{l+s}{\PYZbs{}}\PY{l+s}{d+      \PYZsh{} egész rész}
\PY{l+s}{   }\PY{l+s}{\PYZbs{}}\PY{l+s}{.       \PYZsh{} tizedes pont}
\PY{l+s}{   }\PY{l+s}{\PYZbs{}}\PY{l+s}{d*      \PYZsh{} törtrész}
\PY{l+s}{"""}\PY{p}{,}
\PY{l+s}{"}\PY{l+s}{123.6}\PY{l+s}{"}\PY{p}{,}
\PY{n}{re}\PY{o}{.}\PY{n}{VERBOSE}\PY{p}{)}
\end{Verbatim}
\end{frame}

\begin{frame}[fragile]
    {Python és Vim összehasonlítása}

    A speciális jelentést:
    \begin{itemize}
        \item egyikben sem kell levédeni:\\
            . (pont), *, \verb'[] ^ $ \'
        \item csak Vimben kell levédeni:\\
            \verb!( ) | { ? +!
    \end{itemize}
\end{frame}

\begin{frame}
    {Dokumentációk}

    A reguláris kifejezések az itt leírtaknál nagyobb mélységgel
    rendelkeznek mind a Python nyelvben, mind a Vim szövegszerkesztőben.

    A reguláris kifejezések Pythonbeli használatáról a Python
    dokumentációk között a Regular Expression HOWTO (a Python HOWTOs
    részben) és az re modul dokumentációja (a Library Reference részben)
    ad jó leírást.

    A reguláris kifejezések Vimbeli használatáról a :help regexp
    parancssal lehet többet megtudni.

    Lásd még: regexp\_vim.txt
\end{frame}

\end{document}
