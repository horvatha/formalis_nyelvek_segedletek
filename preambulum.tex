\usepackage[T1]{fontenc}
\usepackage[utf8x]{inputenc}
\usepackage{inputenc}
\usepackage[magyar]{babel}
\usepackage{ucs}
\usepackage{tabularx}
\usepackage{fancyvrb}
\usepackage{graphicx}

\usepackage{tikz}
\usetikzlibrary{arrows}
\tikzset{ %inner sep = 0.5mm,
  %minimum size = 5mm,
  {selected path/.style} ={draw,opacity=.5,line width=3pt,green},
  {hide path/.style} ={selected path,white,opacity=.9},
  {base point/.style} ={circle,draw=blue!60,fill=blue!30,inner sep=0},
  {small point/.style} ={base point, minimum size= 1mm},
  {middle point/.style} ={base point, minimum size= 3mm},
  {large point/.style} ={base point, minimum size= 10mm},
  {future point/.style} ={circle,color=white,draw},
  point/.style ={circle,draw=black!60,fill=black!20},
  {new point/.style} ={point,fill=blue!30,draw=blue!60},
  {edge/.style} ={thick,draw},
  {new edge/.style} ={draw,thick,-, blue!50},
  {box/.style}={rectangle,  minimum height=1cm,
                anchor=south west, draw, rectangle,
                fill=yellow!20,text centered, text width= 4cm},
  >=latex,
  state/.style = {thick, circle, draw, color = black,
                  fill = red!20, minimum size = 6mm},
  finalstate/.style = {state, fill=green!20, double},
  automat/.style = {thick, draw, color = black, fill = #1!12,
                  minimum size = 10mm},
  automat/.default = black,
  point/.style ={circle,draw=blue!60,fill=blue!30,
                  inner sep=0, minimum size= 1mm},
 }
\usepackage{pgfplots}
\usepackage{pgfplotstable}
\newcommand{\film}{
\begin{tikzpicture}[rotate=90,scale=.6]
    \draw[fill=black!70] (0,0) -- (0,1) -- (1,1) -- (1,0) -- cycle;
    \draw[fill=black!30] (0.3,0.1) -- (0.3,.9) -- (.7,.9) -- (.7,.1) -- cycle;
    \foreach \i in {.15,.35,...,.76}{
      \draw[fill=white] (0.1,\i) -- (0.1,\i+.1) -- (.2, \i+.1) -- (0.2,\i) -- cycle;
      \draw[fill=white] (0.8,\i) -- (0.8,\i+.1) -- (.9, \i+.1) -- (0.9,\i) -- cycle;
      }
\end{tikzpicture}
}



\mode<article>
{
  \usepackage{fullpage}
  \usepackage{pgf}
  \usepackage[pdftex,colorlinks=true,linkcolor=blue,urlcolor=blue]{hyperref}
  \usepackage{hyperref}
  \usepackage{fancyvrb}
  \setjobnamebeamerversion{python.beamer}
  \newtheorem{definicio}{defin\'ici\'o}
\newcommand{\lecke}[2][90]{\section{ lecke -- #2}

    Tervezett időtartam #1 perc.

}

\newcommand{\coloruncover}[2]{#2}
}

\mode<presentation>
{
%  \usetheme{Dresden}
%  \usetheme{Warsaw}
  \usetheme{Montpellier}
  \usecolortheme{rose}
%  \usepackage{beamerthemebars}

%  \setbeamercovered{transparent}
  %\setbeamertemplate{headline}{}
  \hypersetup{
  pdftitle={Horvath Arpad: Bevezetes a halozatok vizsgalataba},
  unicode
  }
  \newtheorem{definicio}{Defin\'ici\'o}
\newcommand{\lecke}[2][90]{\section{#2}

\begin{frame}
    {Lecke: #2}

    \vfill
    Tervezett időtartam #1 perc.

    \vfill
\end{frame}
}
\definecolor{uncovercolor}{rgb}{1,0.8,0.8}
\newcommand{\coloruncover}[2]{\temporal<#1>{\colorbox{bg}{\phantom{#2}}}{\colorbox{uncovercolor}{\phantom{#2}}}{\colorbox{bg}{#2}}}

}

\newcommand{\oktato}{\textsc{\bfseries\Large\color{green!70!black}PyOkt} }


\newcommand{\Smiley}[1]{%
\begin{tikzpicture}[scale=0.2]
    \newcommand*{\SmileyRadius}{1.0}%
    \draw [fill=brown!10] (0,0) circle (\SmileyRadius)% outside circle
        %node [yshift=-0.22*\SmileyRadius cm] {\tiny #1}% uncomment this to see the smile factor
        ;  

    \pgfmathsetmacro{\eyeX}{0.5*\SmileyRadius*cos(30)}
    \pgfmathsetmacro{\eyeY}{0.5*\SmileyRadius*sin(30)}
    \draw [fill=cyan,draw=none] (\eyeX,\eyeY) circle (0.15cm);
    \draw [fill=cyan,draw=none] (-\eyeX,\eyeY) circle (0.15cm);

    \pgfmathsetmacro{\xScale}{2*\eyeX/180}
    \pgfmathsetmacro{\yScale}{1.0*\eyeY}
    \draw[color=red, domain=-\eyeX:\eyeX]   
        plot ({\x},{
            -0.1+#1*0.15 % shift the smiley as smile decreases
            -#1*1.75*\yScale*(sin((\x+\eyeX)/\xScale))-\eyeY});
\end{tikzpicture}%
}%

\newcommand{\utolsooldal}[1][.]{
\begin{frame}
    {Gratulálunk!}

    Ezzel a modul leckéit sikeresen teljesítette.

    \vfill
    Ami még modulzáró feladatként hátra van:\\
    a modulzáró teszt megoldása#1
\end{frame}
}
%pygments-hez
\usepackage{fancyvrb}
\usepackage{color}

\makeatletter
\def\PY@reset{\let\PY@it=\relax \let\PY@bf=\relax%
    \let\PY@ul=\relax \let\PY@tc=\relax%
    \let\PY@bc=\relax \let\PY@ff=\relax}
\def\PY@tok#1{\csname PY@tok@#1\endcsname}
\def\PY@toks#1+{\ifx\relax#1\empty\else%
    \PY@tok{#1}\expandafter\PY@toks\fi}
\def\PY@do#1{\PY@bc{\PY@tc{\PY@ul{%
    \PY@it{\PY@bf{\PY@ff{#1}}}}}}}
\def\PY#1#2{\PY@reset\PY@toks#1+\relax+\PY@do{#2}}

\def\PY@tok@gd{\def\PY@tc##1{\textcolor[rgb]{0.63,0.00,0.00}{##1}}}
\def\PY@tok@gu{\let\PY@bf=\textbf\def\PY@tc##1{\textcolor[rgb]{0.50,0.00,0.50}{##1}}}
\def\PY@tok@gt{\def\PY@tc##1{\textcolor[rgb]{0.00,0.25,0.82}{##1}}}
\def\PY@tok@gs{\let\PY@bf=\textbf}
\def\PY@tok@gr{\def\PY@tc##1{\textcolor[rgb]{1.00,0.00,0.00}{##1}}}
\def\PY@tok@cm{\let\PY@it=\textit\def\PY@tc##1{\textcolor[rgb]{0.25,0.50,0.50}{##1}}}
\def\PY@tok@vg{\def\PY@tc##1{\textcolor[rgb]{0.10,0.09,0.49}{##1}}}
\def\PY@tok@m{\def\PY@tc##1{\textcolor[rgb]{0.40,0.40,0.40}{##1}}}
\def\PY@tok@mh{\def\PY@tc##1{\textcolor[rgb]{0.40,0.40,0.40}{##1}}}
\def\PY@tok@go{\def\PY@tc##1{\textcolor[rgb]{0.50,0.50,0.50}{##1}}}
\def\PY@tok@ge{\let\PY@it=\textit}
\def\PY@tok@vc{\def\PY@tc##1{\textcolor[rgb]{0.10,0.09,0.49}{##1}}}
\def\PY@tok@il{\def\PY@tc##1{\textcolor[rgb]{0.40,0.40,0.40}{##1}}}
\def\PY@tok@cs{\let\PY@it=\textit\def\PY@tc##1{\textcolor[rgb]{0.25,0.50,0.50}{##1}}}
\def\PY@tok@cp{\def\PY@tc##1{\textcolor[rgb]{0.74,0.48,0.00}{##1}}}
\def\PY@tok@gi{\def\PY@tc##1{\textcolor[rgb]{0.00,0.63,0.00}{##1}}}
\def\PY@tok@gh{\let\PY@bf=\textbf\def\PY@tc##1{\textcolor[rgb]{0.00,0.00,0.50}{##1}}}
\def\PY@tok@ni{\let\PY@bf=\textbf\def\PY@tc##1{\textcolor[rgb]{0.60,0.60,0.60}{##1}}}
\def\PY@tok@nl{\def\PY@tc##1{\textcolor[rgb]{0.63,0.63,0.00}{##1}}}
\def\PY@tok@nn{\let\PY@bf=\textbf\def\PY@tc##1{\textcolor[rgb]{0.00,0.00,1.00}{##1}}}
\def\PY@tok@no{\def\PY@tc##1{\textcolor[rgb]{0.53,0.00,0.00}{##1}}}
\def\PY@tok@na{\def\PY@tc##1{\textcolor[rgb]{0.49,0.56,0.16}{##1}}}
\def\PY@tok@nb{\def\PY@tc##1{\textcolor[rgb]{0.00,0.50,0.00}{##1}}}
\def\PY@tok@nc{\let\PY@bf=\textbf\def\PY@tc##1{\textcolor[rgb]{0.00,0.00,1.00}{##1}}}
\def\PY@tok@nd{\def\PY@tc##1{\textcolor[rgb]{0.67,0.13,1.00}{##1}}}
\def\PY@tok@ne{\let\PY@bf=\textbf\def\PY@tc##1{\textcolor[rgb]{0.82,0.25,0.23}{##1}}}
\def\PY@tok@nf{\def\PY@tc##1{\textcolor[rgb]{0.00,0.00,1.00}{##1}}}
\def\PY@tok@si{\let\PY@bf=\textbf\def\PY@tc##1{\textcolor[rgb]{0.73,0.40,0.53}{##1}}}
\def\PY@tok@s2{\def\PY@tc##1{\textcolor[rgb]{0.73,0.13,0.13}{##1}}}
\def\PY@tok@vi{\def\PY@tc##1{\textcolor[rgb]{0.10,0.09,0.49}{##1}}}
\def\PY@tok@nt{\let\PY@bf=\textbf\def\PY@tc##1{\textcolor[rgb]{0.00,0.50,0.00}{##1}}}
\def\PY@tok@nv{\def\PY@tc##1{\textcolor[rgb]{0.10,0.09,0.49}{##1}}}
\def\PY@tok@s1{\def\PY@tc##1{\textcolor[rgb]{0.73,0.13,0.13}{##1}}}
\def\PY@tok@sh{\def\PY@tc##1{\textcolor[rgb]{0.73,0.13,0.13}{##1}}}
\def\PY@tok@sc{\def\PY@tc##1{\textcolor[rgb]{0.73,0.13,0.13}{##1}}}
\def\PY@tok@sx{\def\PY@tc##1{\textcolor[rgb]{0.00,0.50,0.00}{##1}}}
\def\PY@tok@bp{\def\PY@tc##1{\textcolor[rgb]{0.00,0.50,0.00}{##1}}}
\def\PY@tok@c1{\let\PY@it=\textit\def\PY@tc##1{\textcolor[rgb]{0.25,0.50,0.50}{##1}}}
\def\PY@tok@kc{\let\PY@bf=\textbf\def\PY@tc##1{\textcolor[rgb]{0.00,0.50,0.00}{##1}}}
\def\PY@tok@c{\let\PY@it=\textit\def\PY@tc##1{\textcolor[rgb]{0.25,0.50,0.50}{##1}}}
\def\PY@tok@mf{\def\PY@tc##1{\textcolor[rgb]{0.40,0.40,0.40}{##1}}}
\def\PY@tok@err{\def\PY@bc##1{\fcolorbox[rgb]{1.00,0.00,0.00}{1,1,1}{##1}}}
\def\PY@tok@kd{\let\PY@bf=\textbf\def\PY@tc##1{\textcolor[rgb]{0.00,0.50,0.00}{##1}}}
\def\PY@tok@ss{\def\PY@tc##1{\textcolor[rgb]{0.10,0.09,0.49}{##1}}}
\def\PY@tok@sr{\def\PY@tc##1{\textcolor[rgb]{0.73,0.40,0.53}{##1}}}
\def\PY@tok@mo{\def\PY@tc##1{\textcolor[rgb]{0.40,0.40,0.40}{##1}}}
\def\PY@tok@kn{\let\PY@bf=\textbf\def\PY@tc##1{\textcolor[rgb]{0.00,0.50,0.00}{##1}}}
\def\PY@tok@mi{\def\PY@tc##1{\textcolor[rgb]{0.40,0.40,0.40}{##1}}}
\def\PY@tok@gp{\let\PY@bf=\textbf\def\PY@tc##1{\textcolor[rgb]{0.00,0.00,0.50}{##1}}}
\def\PY@tok@o{\def\PY@tc##1{\textcolor[rgb]{0.40,0.40,0.40}{##1}}}
\def\PY@tok@kr{\let\PY@bf=\textbf\def\PY@tc##1{\textcolor[rgb]{0.00,0.50,0.00}{##1}}}
\def\PY@tok@s{\def\PY@tc##1{\textcolor[rgb]{0.73,0.13,0.13}{##1}}}
\def\PY@tok@kp{\def\PY@tc##1{\textcolor[rgb]{0.00,0.50,0.00}{##1}}}
\def\PY@tok@w{\def\PY@tc##1{\textcolor[rgb]{0.73,0.73,0.73}{##1}}}
\def\PY@tok@kt{\def\PY@tc##1{\textcolor[rgb]{0.69,0.00,0.25}{##1}}}
\def\PY@tok@ow{\let\PY@bf=\textbf\def\PY@tc##1{\textcolor[rgb]{0.67,0.13,1.00}{##1}}}
\def\PY@tok@sb{\def\PY@tc##1{\textcolor[rgb]{0.73,0.13,0.13}{##1}}}
\def\PY@tok@k{\let\PY@bf=\textbf\def\PY@tc##1{\textcolor[rgb]{0.00,0.50,0.00}{##1}}}
\def\PY@tok@se{\let\PY@bf=\textbf\def\PY@tc##1{\textcolor[rgb]{0.73,0.40,0.13}{##1}}}
\def\PY@tok@sd{\let\PY@it=\textit\def\PY@tc##1{\textcolor[rgb]{0.73,0.13,0.13}{##1}}}

\def\PYZbs{\char`\\}
\def\PYZus{\char`\_}
\def\PYZob{\char`\{}
\def\PYZcb{\char`\}}
\def\PYZca{\char`\^}
\def\PYZsh{\char`\#}
\def\PYZpc{\char`\%}
\def\PYZdl{\char`\$}
\def\PYZti{\char`\~}
% for compatibility with earlier versions
\def\PYZat{@}
\def\PYZlb{[}
\def\PYZrb{]}
\makeatother
% pygments rész vége

\newcommand{\figs}{{cn_edu/pdf}}
\newcommand{\deb}{{\em deb}}
\newcommand{\apt}{{\tt apt}}
\newcommand{\kin}{$k_{\rm in}$}
\newcommand{\kout}{$k_{\rm out}$}
\newcommand{\megoldasjon}{\vfill
\begin{tikzpicture}
    \node[draw,thick,green!50!black,circle,inner sep=1] (A) at (0,0) {M};
\end{tikzpicture}
{\color{red}Megoldás jön!}
Lapozás előtt próbálja megoldani a feladatot!}

\newcommand{\modul}[1]{\section{modul -- #1}}
\newcommand{\modulcel}{\par \textbf{A modul célja:} }
\newenvironment{feladat}{\par \textbf{Feladat:}}{}
\newenvironment{kovetelmenyek}{
    \begin{itemize}\itemsep0pt\parskip0pt}
    {\end{itemize}}

\newcommand{\kilencven}{\textbf{\color{red}90}}

\author{Horváth Árpád <horvath.arpad@amk.uni-obuda.hu>}
\subtitle{\"Osszetett h\'al\'ozatok vizsg\'alata}
\institute[ÓE AMK]{Óbudai Egyetem\\
  Alba Regia Műszaki Kar (AMK)\\
  Székesfehérvár}

% Delete this, if you do not want the table of contents to pop up at
% the beginning of each subsection:
\AtBeginSection[]
{
  \begin{frame}<beamer>
    \frametitle{Vázlat}
    \tableofcontents[currentsection]
  \end{frame}
}

\AtBeginSubsection[]
{
  \begin{frame}<beamer>
    \frametitle{Vázlat}
    \tableofcontents[currentsection,currentsubsection]
  \end{frame}
}

\setbeamertemplate{navigation symbols}{}

\newlength{\maxheight}
\setlength{\maxheight}{0.63\textwidth}


